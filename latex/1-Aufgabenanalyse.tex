\chapter{Aufgabenanalyse}
\label{chap:Aufgabenanalyse}

\section{Analyse}
Es soll das Spiel Nim implementiert werden. Nim ist ein Spiel, welches zu zweit mit
Streichhölzern gespielt wird, welche zu Beginn auf eine feste Anzahl sogenannter Reihen
verteilt sind. Bei diesem Spiel gibt es einige Regeln zu beachten. Zwei Spieler ziehen jeweils
abwechselnd und nehmen jeweils in ihrem Zug mindestens ein oder mehrere Streichhölzer
aus einer der Reihen weg. Die Maximalanzahl der Streichhölzer spielt keine Rolle, jedoch
müssen sie ein und derselben Reihe angehören. Der Spieler, welcher das letzte Streichholz
entnimmt, gewinnt das Spiel.
Bei dieser Implementierung beträgt die maximale Anzahl an Reihen neun, die maximale
Anzahl an Streichhölzern in einer Reihe 10. Demzufolge können maximal 90 Streichhölzer auf
dem Feld liegen. Weiterhin sollen zwei Spielstrategien entwickelt werden. Die Strategie des
Spieler 1 soll wenn möglich immer gewinnen. Spieler 2 hingegen zieht alle Hölzer einer Reihe
weg, sofern nur noch in einer Reihe Hölzer liegen. Ist die Anzahl an belegter Reihen größer
als eins, dann zieht Spieler 2 aus einer zufällig gewählten Reihe eine zufällig ausgewählte
Anzahl an Hölzern weg.
Zu einer gegebenen Startverteilung sollen zehn voneinander unabhängige Spiele
durchgeführt werden und dabei die Anzahl der gewonnenen Spiele für jeden Spieler gezählt
werden. Wenn vorhanden sollen die Züge eines gewonnenen Spiels eines Spielers notiert
werden.
Ziel ist es das Spiel lauffähig zu implementieren und eine möglichst gewinnorientierte
Spielstrategie für Spieler 1 zu entwickeln.

\section{Eingabeformat}
Die Eingabedateien enden auf “.in”. Am Anfang einer Eingabedatei stehen beliebig viele
Kommentarzeilen welche mit einem “\#” beginnen. Es muss jedoch mindestens eine
Kommentarzeile mit einer kurzen Beschreibung existieren. Nach der letzten Kommentarzeile
folgt eine Zeile mit der Startverteilung des Spiels. Gültig sind durch Leerzeichen getrennte
ganze Zahlen größer 0 und kleiner 10. Mindestens eine Zahl und maximal neun Zahlen
müssen gegeben sein.
Beispiel einer gültigen Eingabedatei:
\#IHK Beispiel 1 
3 4 5

\section{Ausgabeformat}
Die Ausgabedateien enden auf “.out” und sind gleichnamig zu den jeweiligen Eingabedateien.
Jede Kommentarzeile aus der Eingabedatei soll eins zu eins übernommen werden. Darauf
folgt die Startverteilung, die prozentuale Verteilung der gewonnenen Spiele sowie die Züge
der von den Spielern gewonnenen Spiele, wenn vorhanden.
Beispiel einer Ausgabedatei (korrespondierend zum Beispiel einer Eingabedatei s.o.)
\#IHK Beispiel 1
Startverteilung: (3,4,5)
Gewonnene Spiele Spieler 1: 100%
Gewonnene Spiele Spieler 2: 0%
Beispiel eines von Spieler 1 gewonnenen Spiels:
Zug 1, Spieler 1 : (3,4,5) >
(1,4,5)
Zug 2, Spieler 2 : (1,4,5) >
(1,4,0)
Zug 3. Spieler 1 : (1,4,0) >
(1,1,0)
Zug 4, Spieler 2 : (1,1,0) >
(0,1,0)
Zug 5, Spieler 1 : (0,1,0) >
(0,0,0)
Beispiel eines von Spieler 2 gewonnenen Spiels:
Nicht vorhanden.

\section{Anforderungen an das Programm}
Das Programm arbeitet nach dem MVCEntwurfsmuster.
Die MainFunktion
delegiert die
Verarbeitung an den Controller. Der Controller liest die Daten ein, und speichert sie in das
Model, bearbeitet diese und gibt das Ergebnis über die View aus.
Es ist dabei notwendig auf Fehler angemessen zu reagieren. Das Programm soll mit einer
aussagekräftigen Meldung beendet werden und nicht abstürzen. Die Robustheit des
Programms wird anhand von Testfällen überprüft.



\section{Sonderfälle}
Durch die Analyse der Aufgabenstellung und des Eingabeformates ergeben sich einige
Startverteilungen die äußerst günstig und ungünstig für Spieler 1 sind:
Günstige Startverteilungen:

\begin{itemize}
\item{Es existiert nur eine Reihe}
\item{Es existieren zwei Reihen mit unterschiedlich vielen Hölzern}
\item{Es existieren drei Reihen, bei denen zwei Reihen gleich viele Hölzer beinhalten}
\item{Es existieren drei Reihen, bei denen mit ein Zug eine Verteilung (1,2,3) erzeugt
werden kann}
\item{Es existiere vier Reihen mit der Verteilung (1,2,3,x) , mit beliebigen x}
\end{itemize}



Ungünstige Startverteilungen:

\begin{itemize}
\item{Es existieren zwei Reihen mit einer geringen, gleichen Anzahl an Hölzern}
\end{itemize}



\section{Fehlerfälle}
Fehler können auftreten wenn die Eingabedatei ein unerwünschtes Format beinhaltet:
\begin{itemize}
\item{Keine Kommentarzeile vorhanden}
\item{Erste Zeile nach der Kommentarzeile enthält nicht nur Ziffern}
\item{Ziffern sind nicht durch ein einfaches Leerzeichen getrennt}
\item{Eine Zahl ist gleich 0}
\item{Es werden mehr als 9 Ziffern aufgelistet}
\item{Es wird keine Ziffer aufgelistet}
\end{itemize}

